\documentclass[spec, och, diploma]{SCWorks}
% параметр - тип обучения - одно из значений:
%    spec     - специальность
%    bachelor - бакалавриат (по умолчанию)
%    master   - магистратура
% параметр - форма обучения - одно из значений:
%    och   - очное (по умолчанию)
%    zaoch - заочное
% параметр - тип работы - одно из значений:
%    referat    - реферат
%    coursework - курсовая работа (по умолчанию)
%    diploma    - дипломная работа
%    pract      - отчет по практике
% параметр - включение шрифта
%    times    - включение шрифта Times New Roman (если установлен)
%               по умолчанию выключен
\usepackage{subfigure}
\usepackage{tikz,pgfplots}
\pgfplotsset{compat=1.5}
\usepackage{float}
\usepackage{multirow}

%\usepackage{titlesec}
\setcounter{secnumdepth}{4}
%\titleformat{\paragraph}
%{\normalfont\normalsize}{\theparagraph}{1em}{}
%\titlespacing*{\paragraph}
%{35.5pt}{3.25ex plus 1ex minus .2ex}{1.5ex plus .2ex}

\titleformat{\paragraph}[block]
{\hspace{1.25cm}\normalfont}
{\theparagraph}{1ex}{}
\titlespacing{\paragraph}
{0cm}{2ex plus 1ex minus .2ex}{.4ex plus.2ex}

% --------------------------------------------------------------------------%

\usepackage[T2A]{fontenc}
\usepackage[utf8]{inputenc}
\usepackage{graphicx}
\graphicspath{ {./images/} }
\usepackage{tempora}

\usepackage[sort,compress]{cite}
\usepackage{amsmath}
\usepackage{amssymb}
\usepackage{amsthm}
\usepackage{fancyvrb}
\usepackage{listings}
\usepackage{listingsutf8}
\usepackage{longtable}
\usepackage{array}
\usepackage[english,russian]{babel}

% \usepackage[colorlinks=true]{hyperref}
\usepackage{url}

\usepackage{underscore}
\usepackage{setspace}
\usepackage{indentfirst} 
\usepackage{mathtools}
\usepackage{amsfonts}
\usepackage{enumitem}
\usepackage{tikz}
\usepackage{minted}

\setminted[py]{fontsize=\small, breaklines=true, style=bw, linenos}

\newcommand{\eqdef}{\stackrel {\rm def}{=}}
\newcommand{\specialcell}[2][c]{%
\begin{tabular}[#1]{@{}c@{}}#2\end{tabular}}

\renewcommand\theFancyVerbLine{\small\arabic{FancyVerbLine}}

\newtheorem{lem}{Лемма}

\begin{document}

% Кафедра (в родительном падеже)
\chair{теоретических основ компьютерной безопасности и криптографии}

% Тема работы
\title{Обнаружение аномалий в мультивариативных временных рядах методами искусственного
интеллекта}

% Курс
\course{5}

% Группа
\group{531}

% Факультет (в родительном падеже) (по умолчанию "факультета КНиИТ")
\department{факультета КНиИТ}

% Специальность/направление код - наименование
%\napravlenie{09.03.04 "--- Программная инженерия}
%\napravlenie{010500 "--- Математическое обеспечение и администрирование информационных систем}
%\napravlenie{230100 "--- Информатика и вычислительная техника}
%\napravlenie{231000 "--- Программная инженерия}
\napravlenie{100501 "--- Компьютерная безопасность}

% Для студентки. Для работы студента следующая команда не нужна.
% \studenttitle{Студентки}

% Фамилия, имя, отчество в родительном падеже
\author{Улитина Ивана Владимировича}

% Заведующий кафедрой
\chtitle{} % степень, звание
\chname{Абросимов М. Б.}

%Научный руководитель (для реферата преподаватель проверяющий работу)
\satitle{д.ф.-м.н., доцент} %должность, степень, звание
\saname{Слеповичев И. И.}

% Руководитель практики от организации (только для практики,
% для остальных типов работ не используется)
% \patitle{к.ф.-м.н.}
% \paname{С.~В.~Миронов}

% Семестр (только для практики, для остальных
% типов работ не используется)
%\term{8}

% Наименование практики (только для практики, для остальных
% типов работ не используется)
%\practtype{преддипломная}

% Продолжительность практики (количество недель) (только для практики,
% для остальных типов работ не используется)
%\duration{4}

% Даты начала и окончания практики (только для практики, для остальных
% типов работ не используется)
%\practStart{30.04.2019}
%\practFinish{27.05.2019}

% Год выполнения отчета
\date{2024}

\maketitle

% Включение нумерации рисунков, формул и таблиц по разделам
% (по умолчанию - нумерация сквозная)
% (допускается оба вида нумерации)
% \secNumbering

%-------------------------------------------------------------------------------------------

\tableofcontents

\intro

    \textbf{Временным рядом} называют последовательность значений, упорядоченных
    по времени, в которое то или иное значение было получено/зафиксировано. В
    качестве примеров временного ряда можно выделить:

    \begin{enumerate}
        \item курсы валют/акций (стоимость валюты/акции в конкретный момент
        времени);
        \item значение прогноза погоды, параметры работы двигателя (изменяющееся
        с течением времени значение некоторой физической величины);
        \item электрокардиограмма (биологический параметр человека);
        \item передаваемый объём сетевого трафика.
    \end{enumerate}

    Временной ряд обладает типовыми характеристиками (иногда называемые
    компонентами), которые точно описывают характер временного ряда и в виде
    совокупности которых временной ряд можно представить:

    \begin{enumerate}
        \item тренд (долгосрочное увеличение или уменьшение значений ряда);
        \item сезонность, сезонная вариация (краткосрочное регулярно
        повторяющееся колебание значений временного ряда вокруг тренда);
        \item цикл, циклические колебания (характерные изменения ряда, связанные
        с повторяющимися глобальными причинами "--- цикл деловой активности или
        экономический цикл, состоящий из экономического подъема, спада,
        депрессии и оживления);
        \item остаточная вариация, которая может быть двух видов:
            \begin{itemize}
                \item аномальная вариация — неестественно большое отклонение
                временного ряда, которое оказывает воздействие на единичное
                наблюдение;
                \item случайная вариация — малое отклонение, которое невозможно
                предвидеть.
            \end{itemize}
    \end{enumerate}

    Вследствие развития нейросетей и искусственного интеллекта, появилась
    возможность решать задачи, связанные с временными рядами, решение которых
    могло бы быть полезно для той или иной области жизнедеятельности человека.
    Особенно полезно это может быть в тех областях, где обычные статистические
    модели (обычная или интегрированная модель авторегрессии - скользящего
    среднего, векторная авторегрессия и т.д.) неприменимы или неэффективны.

    Временные ряды могут содержать аномалии. \textbf{Аномалией} называется
    отклонение в стандартном поведении какого-то процесса, который описывается
    этим временным рядом. По сути её можно определить через аномальную вариацию,
    являющейся возможной компонентой этого временного ряда. В зависимости от
    предметной области описываемого процесса в выборке (датасете) могут быть
    аномалии разного вида. Можно выделить одни из самых распространенных видов
    аномалий:

    \begin{enumerate}
        \item точечные аномалии (наблюдается отклонение в поведении в отдельных
        точках);
        \item групповые аномалии (группа точек, которая ведет себя аномально, но
        каждая точка которой отдельно аномальной не является);
        \item контекстные аномалии, суть которых в связи аномалии с внешними
        данными, которые не присущи значениям ряда (например, отрицательная
        температура на улице летом).
    \end{enumerate}

    \textbf{Мультивариативный временной ряд} представляет собой временной ряд, в
    котором одновременно записывается несколько переменных, например:

    \begin{enumerate}
        \item Измерения метеостанций: температура, влажность, атмосферное
        давление;
        \item Параметры работы промышленного оборудования: ток, напряжение,
        вибрация;
        \item Данные спутников: значения электрического или магнитного поля на
        разных частотах.
    \end{enumerate}

    Обнаружение аномалий в таких рядах особенно важно в ряде практических задач:
    предсказание отказов оборудования на основании датчиков в сфере
    промышленности, обнаружение патологий в данных ЭКГ или ЭЭГ в сфере медицины,
    выявление подозрительных транзакций или внезапных рыночных колебаний в сфере
    анализа финансов, выявление редких или необычных событий наподобие всплесков
    радиации в сфере астрономии.

    В данной работе исследуется возможность использования классических
    алгоритмов машинного обучения (случайный лес, метод k ближайших соседей,
    CatBoost) и нейросетевых моделей компьютерного зрения (ResNet34, Xception,
    ViT) для классификации мультивариативных временных рядов при решении задачи
    обнаружения аномалий. В качестве набора данных используются данные,
    полученные со спутника GGS WIND. Это 256 временных рядов, которые
    представляют собой значения нормализованного напряжения электрического поля,
    каждый из которых соответствует определенной частоте.
    
    В качестве предобработки данных были использованы два подхода:
    \begin{enumerate}
        \item Метод скользящего окна "--- для получения временных блоков
        фиксированной длительности;
        \item Метод преобразования Фурье для построения спектрограмм.
    \end{enumerate}
    
    Классические алгоритмы машинного обучения в качестве выборки использовали
    временные блоки, полученные первым методом, а модели компьютерного зрения -
    спектрограммы, полученные вторым методом. Сравнение результатов моделей
    проводилось по метрикам точности классификации.

\section{Теоретическая часть}

    \subsection{Временных ряды в задачах машинного обучения и их обработка}
    
        \subsubsection{Временных ряды в задачах машинного обучения}

            Временной ряд "--- это упорядоченная последовательность данных,
            измеряемых через равные промежутки времени. Основной характеристикой
            временного ряда является наличие зависимости между наблюдениями,
            сделанными в различные моменты времени. Например, измерения температуры
            в течение суток, записи электрокардиограммы (ЭКГ) или значения продаж
            товаров по дням.

            Мультивариативные временные ряды представляют собой более сложную форму
            временных рядов, в которых одновременно записывается несколько
            переменных. Эти ряды позволяют изучать взаимосвязи между различными
            параметрами системы, такие как изменения температуры, влажности и
            давления в метеорологии или параметры работы двигателя (температура,
            вибрация, ток).

            Временные ряды являются важным источником информации в различных
            областях. Их анализ помогает решать широкий круг задач:

            \begin{enumerate}
                \item Прогнозирование "--- предсказание будущих значений временного
                ряда, например, прогнозирование спроса на продукцию, движения
                фондового рынка или погоды;
                \item Обнаружение аномалий "--- выявление отклонений от нормального
                поведения, таких как сбои оборудования, резкие изменения в сердечной
                активности или финансовые мошенничества;
                \item Классификация "--- определение категории или состояния на
                основе временного ряда, например, диагностика заболеваний по данным
                ЭКГ;
                \item Кластеризация "--- группировка временных рядов с похожими
                характеристиками, например, выявление схожих паттернов в потреблении
                электроэнергии.
            \end{enumerate}

            Работа с временными рядами накладывает ряд специфических требований:

            \begin{enumerate}
                \item Наблюдения в ряде зависят от предшествующих значений;
                Например, в данных о погоде температура в текущий момент связана с
                температурой в предыдущие часы;
                \item Временные ряды часто имеют большую длину, а мультивариативные
                ряды содержат множество переменных, что требует мощных
                вычислительных ресурсов;
                \item Данные могут быть зашумлены, содержать пропуски или выбросы,
                что делает их предварительную обработку важным этапом;
                \item Временные ряды могут содержать как линейные, так и нелинейные
                зависимости, а также периодические и непериодические компоненты.
            \end{enumerate}

        \subsubsection{Обработка временных рядов}

            Обработка временных рядов представляет собой ключевой этап
            подготовки данных для последующего анализа и построения моделей
            машинного обучения. Временные ряды обладают специфическими
            особенностями, такими как наличие временной зависимости, высокая
            размерность, наличие шума и выбросов, которые требуют применения
            специализированных методов. Эффективная обработка временных рядов
            позволяет выделить скрытые закономерности, устранить помехи и
            подготовить данные для анализа.

            Для успешного применения методов машинного обучения временные ряды
            часто преобразуются из исходного формата в более удобное для анализа
            представление. Среди наиболее распространенных методов
            преобразования временных рядов выделяются:

            \begin{enumerate}
                \item Быстрое преобразование Фурье (FFT)
            
                    Этот метод используется для перехода от временной области к
                    частотной. На его основе строятся спектры амплитуд и
                    мощностей, которые дают представление о частотных
                    компонентах сигнала. Например, частотный анализ позволяет
                    выделить доминирующие частоты, связанные с определенными
                    процессами, что полезно для диагностики технических систем
                    или биологических сигналов.

                    Быстрое преобразование Фурье (Fast Fourier Transform, FFT) —
                    это алгоритм вычисления дискретного преобразования Фурье
                    (DFT) и его обратного преобразования (IDFT). Оно позволяет
                    анализировать временной сигнал, переводя его из временной
                    области в частотную, чтобы определить амплитуды и фазы
                    составляющих сигнал частот.

                    Для сигнала $x[n]$, содержащего $N$ отсчетов, дискретное
                    преобразование Фурье определяется следующим образом:

                    \[X[k] = \sum_{n=0}^{N - 1} x[n] e^{-j2\pi\frac{kn}{N}}, \text{ } k = 0, 1, \dots, N - 1,\]

                    где $X[k]$ "--- комплексное число, представляющее амплитуду
                    и фазу частоты $k$, $x[n]$ "--- значение сигнала во
                    временной области в момент $n$, $N$ "--- длина временного
                    ряда, $e^{-j2\pi\frac{kn}{N}}$ "---  комплексный
                    экспоненциальный множитель (база Фурье).

                    Прямое вычисление DFT имеет сложность $O(N^2)$, так как для
                    каждого $k$ необходимо выполнить $N$ умножений и сложений.
                    Это делает применение DFT к длинным временным рядам
                    вычислительно затратным. FFT — это оптимизация алгоритма
                    DFT, которая уменьшает его сложность до $O(NlogN)$. Алгоритм
                    FFT основан на принципе ''разделяй и властвуй'', разбивая
                    исходную задачу на меньшие подзадачи.

                    Для построения спектрограммы применяется FFT к небольшим
                    сегментам временного ряда (окон). Это позволяет отобразить
                    изменение частотных составляющих сигнала во времени. Выбор
                    длины окна и перекрытия определяет разрешение по времени и
                    частоте.

                \item Вейвлет-преобразование

                    Вейвлеты позволяют анализировать временной ряд на разных
                    временных масштабах, что особенно важно для данных,
                    содержащих локальные особенности или резкие изменения. Этот
                    метод часто используется для выделения кратковременных
                    событий, таких как скачки напряжения в энергетических
                    системах или аномальные всплески в радиосигналах.

                \item Создание лаговых признаков
                
                    Для сохранения временной зависимости между наблюдениями в
                    ряде, создаются дополнительные признаки, представляющие
                    собой значения ряда на предыдущих шагах. Например, для
                    предсказания температуры в момент времени $t$, могут быть
                    использованы значения температуры в $t-1$, $t-2$ и так
                    далее. Такой подход сохраняет информацию о динамике ряда,
                    что важно для задач прогнозирования.

                \item Сглаживание, нормализация и устранение выбросов;

            \end{enumerate}




    \subsection{Классические алгоритмы машинного обучения}

        \subsubsection{Случайный лес}

        \subsubsection{Метод k случайных соседей}
        
        \subsubsection{Boosting-алгоритм CatBoost}

    \subsection{Алгоритмы компьютерного зрения}

        \subsubsection{ResNet34}

        \subsubsection{Xception}
        
        \subsubsection{ViT}

    \subsection{Метрики оценки качества обучения}

\section{Практическая часть}

    \subsection{Набор данных}

        В данной работе в качестве мультивариативного временного ряда
        использовались данные спутника GGS WIND, которые представляют собой
        значения напряженности электрического поля для частот от 20 кГц до 13825
        кГц, измеряемые раз в минуту на протяжении 18720 минут. Измерения были
        зафиксированы в период времени с 1 июля 2020 года в 00:00:30 по 13 июля
        2020 года в 23:59:30. Для каждой частоты в диапазоне с шагом 4 кГц
        соответствовал отдельный временной ряд, фиксирующий напряженность
        электрического поля в конкретный момент времени.
        
        Таким образом, полученный мультивариативный временной ряд состоит из 256
        временных рядов, разделенных на интервалы, где фиксировалось наличие или
        отсутствие аврорального километрового излучения. Авроральное
        километровое радиоизлучение (АКР) — это интенсивное природное
        радиоизлучение в диапазоне частот 30–800 кГц, создаваемое в околоземной
        плазме и распространяющееся от Земли.

    \subsection{Обработка набора данных и создание спектрограмм}

    \subsection{Программная реализация}

    \subsection{Результаты обучения}

\conclusion


\begin{thebibliography}{99}
    % \bibitem{mediasphera} Белоусов, К. И. Когнитивно-информационное
    % моделирование социальной реальности: концепты, события, приоритеты / К. И.
    % Белоусов, Д. А. Баранов, Н. Л. Зелянская, Н. Ф. Пономарев, К. В. Рябинин //
    % Вестник Томского гос. ун-та. Филология. "--- 2021. "--- №72. "--- С. 5-26.
    
    % \bibitem{w2v} Mikolov, T. Efficient estimation of word representations in
    % vector space. / T. Mikolov, K. Chen, G. Corrado, J. Dean "--- 2013. "---
    % arXiv preprint. "--- arXiv:1301.3781.
    
    % \bibitem{fasttext} Bojanowski, P. Enriching word vectors with subword
    % information. / P. Bojanowski, E. Grave, A. Joulin, T. Mikolov //
    % Transactions of the association for computational linguistics. "--- 2017.
    % "--- Vol. 5. "--- Pp. 135-146.
    
    % \bibitem{deepwalk} Perozzi, B. Deepwalk: Online learning of social
    % representations / B. Perozzi, R. Al-Rfou, S. Skiena // In Proceedings of the
    % 20th ACM SIGKDD international conference on Knowledge discovery and data
    % mining. "--- 2014. "--- Pp. 701-710.

    % \bibitem{pca} Abdi, H. Principal component analysis / H. Abdi, L.J. Williams
    % // Wiley interdisciplinary reviews: computational statistics. "--- 2010.
    % "--- Vol. 2. "--- Pp. 433-459.

    % \bibitem{silhouette} Rousseeuw, P.J. Silhouettes: a graphical aid to the
    % interpretation and validation of cluster analysis / P.J. Rousseeuw //
    % Journal of computational and applied mathematics. "--- 1987. "--- Vol. 20.
    % "--- Pp. 53-65.

    % \bibitem{kmeans} Ahmed, M. The k-means algorithm: A comprehensive survey and
    % performance evaluation / M. Ahmed, R. Seraj, S.M.S. Islam // Electronics.
    % "--- 2020. "--- Vol. 9. "--- Pp. 1295.

    % \bibitem{concepts1} Вежбицкая, А. Семантика, культура и познание:
    % общечеловеческие понятия в культуроспецифичных контекстах. / А. Вежбицкая //
    % Thesis. "--- 1993. "--- №3. "--- С. 185.

    % \bibitem{concepts2} Степанов, Ю. Константы. Словарь русской культуры. / Ю.
    % Степанов // Опыт исследования. "--- 2017.

    % \bibitem{resume} Белоусов, К. И. Профилирование концептуальных систем на
    % основев комплекса методов психосемантики и машинного обучения / К. И.
    % Белоусов, Р. К. Баширов, Н. Л. Зелянская, И. А. Лабутин, К. В. Рябинин, Р.
    % В. Чумаков // Научно-техническая информация. "--- 2023. "--- №7. "--- С.
    % 1-14.

\end{thebibliography}

\appendix

    % \section{Листинг \texttt{preprocessing.py}}
    % \inputminted{py}{code/preprocessing.py}

    % \section{Листинг \texttt{models.py}}
    % \inputminted{py}{code/models.py}

    % \section{Листинг \texttt{main.py}}
    % \inputminted{py}{code/main.py}

\end{document}
